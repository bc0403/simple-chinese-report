% 简洁中文学术报告模板
% 基于浙江大学 zjuthesis 模板简化而来
% 适用于简单的中文学术报告、课程论文等
%
% 编译说明:
% 方法1: 使用 compile-simple.sh (Linux/Mac) 或 compile-simple.bat (Windows)
% 方法2: 手动编译: xelatex -> biber -> xelatex -> xelatex
% 方法3: 使用 latexmk: latexmk -pdf -xelatex simple-chinese-report.tex
% 方法4: 使用自定义 latexmk 配置: latexmk -r simple-latexmkrc simple-chinese-report.tex

\documentclass[UTF8, zihao=-4]{ctexart}

% 页面设置
\usepackage[top=2.5cm, bottom=2.5cm, left=3cm, right=2.5cm]{geometry}
\usepackage{fancyhdr}
\usepackage{titlesec}
\usepackage{graphicx}
\usepackage{booktabs}
\usepackage{amsmath}
\usepackage{amssymb}
\usepackage{hyperref}
\usepackage{enumitem}
\usepackage{caption}

% 设置段落间距和缩进
\setlength{\parskip}{0.5em}
\setlength{\parindent}{2em}

% 设置章节格式
\ctexset{
    section = {
        format = \zihao{-3}\bfseries,
        nameformat = {},
        aftername = \quad,
        beforeskip = 1.0ex plus 0.2ex minus .2ex,
        afterskip = 1.0ex plus 0.2ex minus .2ex,
    },
    subsection = {
        format = \zihao{4}\bfseries,
        nameformat = {},
        aftername = \quad,
        beforeskip = 1.0ex plus 0.2ex minus .2ex,
        afterskip = 1.0ex plus 0.2ex minus .2ex,
    },
    subsubsection = {
        format = \zihao{-4}\bfseries,
        nameformat = {},
        aftername = \quad,
        beforeskip = 1.0ex plus 0.2ex minus .2ex,
        afterskip = 1.0ex plus 0.2ex minus .2ex,
    }
}

% 设置页眉页脚
\pagestyle{fancy}
\fancyhf{}
\fancyhead[C]{\zihao{-5}\leftmark}
\fancyfoot[C]{\zihao{-5}\thepage}
\renewcommand{\headrulewidth}{0.4pt}

% 设置图表标题格式
\captionsetup[figure]{
    font=small,
    labelfont=bf,
    labelsep=space
}
\captionsetup[table]{
    font=small,
    labelfont=bf,
    labelsep=space
}

% 自定义命令
\newcommand{\keywords}[1]{\textbf{关键词:}#1}
\newcommand{\abstracttext}[1]{\textbf{摘要:}#1}

\begin{document}

% 标题页
\begin{titlepage}
    \centering
    \vspace*{2cm}

    {\zihao{1}\bfseries 报告标题\\}
    \vspace{1cm}

    {\zihao{3} 副标题(可选)\\}
    \vspace{3cm}

    {\zihao{4}
    \begin{tabular}{rl}
        作者: & 姓名 \\
        学号: & 12345678 \\
        专业: & 专业名称 \\
        学院: & 学院名称 \\
        指导教师: & 教师姓名 \\
        日期: & \today
    \end{tabular}
    }
    \vfill

    {\zihao{-4} 学校名称\\}
    \vspace{1cm}
\end{titlepage}

% 摘要页
\newpage
\thispagestyle{empty}
\section*{摘要}
\abstracttext{
    这里是摘要内容。摘要应该简洁明了地概括报告的主要内容、研究方法、主要发现和结论。
    摘要通常在200-300字之间,能够让读者快速了解报告的核心内容。
}

\vspace{1em}
\keywords{关键词1;关键词2;关键词3}

\newpage
\tableofcontents
\newpage

% 正文开始
\section{引言}

这里是引言部分。引言应该介绍研究背景、研究意义、国内外研究现状以及本文的研究内容和结构安排。

近年来,人工智能技术在各个领域取得了显著进展\cite{lecun2015deep}。特别是在自然语言处理方面,深度学习模型\cite{vaswani2017attention}已经能够处理复杂的语言任务。这些技术的发展为本研究提供了重要的理论基础和技术支持。

\subsection{研究背景}

研究背景部分应该说明为什么选择这个研究课题,以及该课题在当前学术领域或实际应用中的重要性。

\subsection{研究意义}

研究意义可以从理论意义和实践意义两个方面进行阐述。

\section{相关工作}

相关工作部分应该综述与本研究相关的已有研究成果,分析现有研究的优缺点,并指出本研究的创新点。

在相关研究领域,已有许多学者进行了深入探索。例如,在机器学习领域,\cite{goodfellow2016deep} 系统性地介绍了深度学习的基本原理和应用。同时,在中文自然语言处理方面,\cite{liu2019roberta} 提出的中文预训练模型在多个任务上取得了优异表现。

\subsection{国内外研究现状}

可以分别介绍国内外的研究现状,并进行对比分析。

\section{研究方法}

研究方法部分应该详细说明研究所采用的方法、技术路线、实验设计等。

\subsection{理论框架}

如果有理论框架,可以在这里进行说明。

\subsection{实验设计}

实验设计部分应该包括实验对象、实验设备、实验步骤等内容。

\section{结果与分析}

结果与分析部分应该展示研究结果,并对结果进行详细的分析和讨论。

\subsection{实验结果}

可以在这里展示实验数据、图表等。

\begin{figure}[htbp]
    \centering
    \includegraphics[width=0.6\textwidth]{example-image}
    \caption{示例图片}
    \label{fig:example}
\end{figure}

\subsection{数据分析}

对实验结果进行统计分析,验证假设或得出结论。

\begin{table}[htbp]
    \centering
    \caption{示例表格}
    \label{tab:example}
    \begin{tabular}{ccc}
        \toprule
        项目 & 数值1 & 数值2 \\
        \midrule
        实验组 & 10.5 & 20.3 \\
        对照组 & 8.2 & 18.7 \\
        \bottomrule
    \end{tabular}
\end{table}

\section{结论}

结论部分应该总结全文的主要内容和研究成果,指出研究的局限性,并提出未来研究的方向。

\subsection{研究总结}

简要总结研究的主要内容和重要发现。

\subsection{研究展望}

提出未来可以进一步研究的方向和问题。

\begin{thebibliography}{99}
\bibitem{lecun2015deep} LeCun Y, Bengio Y, Hinton G. Deep learning[J]. Nature, 2015, 521(7553): 436-444.
\bibitem{vaswani2017attention} Vaswani A, Shazeer N, Parmar N, et al. Attention is all you need[C]//Advances in Neural Information Processing Systems. 2017: 5998-6008.
\bibitem{goodfellow2016deep} Goodfellow I, Bengio Y, Courville A. Deep learning[M]. MIT press, 2016.
\bibitem{liu2019roberta} Liu Y, Ott M, Goyal N, et al. Roberta: A robustly optimized bert pretraining approach[J]. arXiv preprint arXiv:1907.11692, 2019.
\end{thebibliography}

\newpage
\section*{致谢}

感谢所有对本文研究工作给予支持和帮助的个人和机构。

感谢指导教师的悉心指导,感谢实验室同学的帮助,感谢家人的支持。

\end{document}